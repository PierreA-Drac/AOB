% Preamble
% =============================================================================

% Class of the document.
\documentclass[12pt,a4paper]{article}
% article : short article.
% report  : mid-length report.
% book    : book or thesis redaction.

% Paragraph skip length (default to 0).
\setlength{\parskip}{1ex}

% Packages
% =============================================================================

% Encoding
% -----------------------------------------------------------------------------

% Babel.
\usepackage[french]{babel}
% FontEnc.
\usepackage[T1]{fontenc}
% InputEnc.
\usepackage[utf8]{inputenc}

% Define \escapeus command to escape underscores.
\makeatletter
\DeclareRobustCommand*{\escapeus}[1]{
    \begingroup\@activeus\scantokens{#1\endinput}\endgroup}
\begingroup\lccode`\~=`\_\relax
    \lowercase{\endgroup\def\@activeus{\catcode`\_=\active \let~\_}}
\makeatother

% Text
% -----------------------------------------------------------------------------

% Acronym.
\usepackage{acronym}
% CsQuote.
\usepackage[style=french,french=guillemets]{csquotes}
% Enumerate.
\usepackage{enumerate}
% HyperRef.
\usepackage[hyperfootnotes=false,hidelinks]{hyperref}
% URL.
\usepackage{url}

% Algorithms
% -----------------------------------------------------------------------------

% Algorithm2E.
\usepackage[french,onelanguage,linesnumbered,ruled,vlined,commentsnumbered]{algorithm2e}

% Source code
% -----------------------------------------------------------------------------

% Listings.
\usepackage{listings}
% Minted.
\usepackage{minted}
% Caption.
\usepackage{caption}
\newenvironment{code}{\captionsetup{type=listing}}{}

% Files
% -----------------------------------------------------------------------------

% FancyVRB.
\usepackage{fancyvrb}
% Redefine \VerbatimInput.
\RecustomVerbatimCommand{\VerbatimInput}{VerbatimInput}
{
    fontsize=\footnotesize,
    frame=lines,         % Top and bottom rule only.
    framesep=1.5em,      % Separation between frame and text.
    rulecolor=\color{red!50!green!50!blue!50!},
    labelposition=topline,
    commandchars=\|\(\), % Escape character and argument delimiters for commands within the verbatim.
    commentchar=*        % Comment character.
}

% Figures
% -----------------------------------------------------------------------------

% GraphicX.
\usepackage{graphicx}
% SVG.
\usepackage{svg}
% WrapFig.
\usepackage{wrapfig}

% Charts
% -----------------------------------------------------------------------------

% PGFPLots
\usepackage{pgfplots}
\pgfplotsset{compat=1.16}
\usepgfplotslibrary{units}

% Mathematics
% -----------------------------------------------------------------------------

% AmsFonts.
\usepackage{amsfonts}
% AmsMath.
\usepackage{amsmath}
% AmsText.
\usepackage{amstext}
% AmsThm.
\usepackage{amsthm}
\newtheorem{prr}{Propriété}
\newtheorem{pro}{Proposition}
\newtheorem{thm}{Théorème}
\newtheorem{lem}{Lemme}
% NumPrint.
\usepackage{numprint}

% Physics
% -----------------------------------------------------------------------------

% Physics.
\usepackage{physics}

% Presentation
% -----------------------------------------------------------------------------

% XColor.
\usepackage{xcolor}

% References
% -----------------------------------------------------------------------------

% CleveRef.
\usepackage{cleveref}

% Structure.
% -----------------------------------------------------------------------------

% Geometry.
\usepackage{geometry}
% PDFLScape.
\usepackage{pdflscape}
% MultiCol.
\usepackage{multicol}
% TitleSec.
\usepackage{titlesec}
\newcommand{\sectionbreak}{\clearpage} % Use a page break before new sections.
% VMargin.
\usepackage{vmargin}
% FootMisc.
\usepackage[bottom]{footmisc}

% Symbols
% -----------------------------------------------------------------------------

% SIUnitX.
\usepackage{siunitx}

% Table
% -----------------------------------------------------------------------------

% Array.
\usepackage{array}
% BookTabs.
\usepackage{booktabs}
% CSVSimple.
\usepackage{csvsimple}

% Document
% =============================================================================

\begin{document}

\title{Analyse de performance et optimisation de code}
\author{Pierre AYOUB}

\maketitle

\begin{figure}[b]
    \centering
    \includegraphics[scale=0.3]{figures/isty.jpg}
\end{figure}

\newpage
\begin{abstract}

La simulation numérique est un procédé informatique visant à modéliser un
phénomène par ordinateur, s’agissant le plus souvent d’un phénomène
physique. Cette modélisation prend forme par des systèmes d’équations
décrivant l’état du système physique représenté à chaque instant. De
nombreux domaines scientifiques convergent vers la simulation
informatique, tel que certaines branches de la physique, de l’analyse
et de l’optimisation mathématique, ou encore le calcul haute
performance en informatique. Enfin, la simulation trouve naturellement
de nombreuses applications concernant des sujets variés, tel que la
simulation du climat et des évènements météorologiques, la simulation
d’essais nucléaires, de l’effet d’un médicament sur un corps, ou encore
des astres et de l’univers. Ce rapport s’articulera donc autour de la
simulation de fumée, phénomène impliquant les lois de la mécanique des
fluides. Notre travail portera sur l’aspect du calcul haute performance
de cette simulation.

\end{abstract}

\tableofcontents

\section{Introduction}
\label{sec.intro}

Le projet que nous vous présentons aujourd’hui consiste à analyser puis,
grâce à nos mesures, optimiser un code de simulation numérique. Ce dernier
nous offre une interface graphique permettant d’ajouter de la fumée dans un
espace confiné et, ainsi, d’en observer le comportement. Nous pouvons
influencer la quantité de fumée et sa vélocité dans l’espace. De plus,
l’application nous donne le contrôle sur la résolution de la simulation,
cela revient à dire sur sa précision, qui détermine principalement la
performance du programme.

Le déroulement du projet s’est effectué en plusieurs étapes distinctes :
\begin{description}
    \item[Analyse du code] Cette phase consiste à analyser le programme
        d’un point de vue mathématique et informatique. De cette première
        approche, il s’agira de comprendre les opérations du programme sur
        les équations qui régissent le système physique. De l’autre
        approche, il convient d’étudier l’architecture logicielle de
        l’application, ainsi que les choix mis en œuvres afin d’implémenter
        le ou les algorithmes nécessaires.
    \item[Protocole expérimental] Une fois l’analyse effectuée, nous
        pouvons en déduire le moyen le plus adapté afin de mesurer les
        performances de notre implémentation. Nous allons donc mettre en
        avant les critères théoriques à atteindre dans nos mesures, puis
        nous exposerons la manière dont nous avons mis ceci en pratique.
    \item[Optimisations et mesures] Grâce au protocole mis en place, nous
        pouvons quantifier la performance du programme. De ce fait, serons
        en mesure d’expérimenter différentes techniques d’optimisation sur
        le programme et d’en calculer l’accélération.
\end{description}

\section{Analyse du code}
\label{sec.analyze}

\VerbatimInput[label=\fbox{\color{red!85!green!85!blue!85!}fluid.cflow},]{callgraph/fluid.cflow}

\begin{figure}[h]
    \centering
    {
        \scriptsize
        \escapeus{\includesvg[scale=0.5]{callgraph/fluid.svg}}
    }
    \caption{Graphe d'appel du fichier \textit{fluid.c}}
    \label{fig.analyze.callgraph}
\end{figure}

\section{Protocole expérimental}
\label{sec.prot}

\subsection{Théorie}
\label{sub.prot.theo}

\subsection{Pratique}
\label{sub.prot.pract}

\section{Optimisations et mesures}
\label{sec.optim}

\subsection{Déroulage de boucle}
\label{sub.optim.unrol}

\subsection{Vectorisation}
\label{sub.optim.vec}

\subsection{Inlining}
\label{sub.optim.inlin}

\section{Conclusion}
\label{sec.conc}

\newpage
\section*{Acronymes}
\label{sec.acro}

\begin{acronym}
    \acro{CPU}  {Central Processing Unit\acroextra{, processeur central de l'ordinateur}}
    \acro{RAM}  {Random Access Memory}
    \acro{HT}   {Hyper-Threading}
\end{acronym}

\end{document}
